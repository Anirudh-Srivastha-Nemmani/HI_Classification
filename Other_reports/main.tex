%%%%%%%%%%%%%%%%%%%%%%%%%%%%%%%%%%%%%%%%%
% Lachaise Assignment
% LaTeX Template
% Version 1.0 (26/6/2018)
%
% This template originates from:
% http://www.LaTeXTemplates.com
%
% Authors:
% Marion Lachaise & François Févotte
% Vel (vel@LaTeXTemplates.com)
%
% License:
% CC BY-NC-SA 3.0 (http://creativecommons.org/licenses/by-nc-sa/3.0/)
% 
%%%%%%%%%%%%%%%%%%%%%%%%%%%%%%%%%%%%%%%%%

%----------------------------------------------------------------------------------------
%	PACKAGES AND OTHER DOCUMENT CONFIGURATIONS
%----------------------------------------------------------------------------------------

\documentclass{article}

\input{structure.tex} % Include the file specifying the document structure and custom commands

%----------------------------------------------------------------------------------------
%	ASSIGNMENT INFORMATION
%----------------------------------------------------------------------------------------

\title{Results} % Title of the assignment

\author{Anirudh}%\\ \texttt{y.amagi@inabauniversity.jp}} % Author name and email address

%\date{University of Inaba --- \today} % University, school and/or department name(s) and a date

%----------------------------------------------------------------------------------------

\begin{document}

\maketitle % Print the title

%----------------------------------------------------------------------------------------
%	INTRODUCTION
%----------------------------------------------------------------------------------------

\section*{Redshift Range} % Unnumbered section

Most of the Associated spectra we have has a redshift range of $ 0.02 < z < 0.25$ as the most of the associated absorbers we had are from Maccagni et al. 2017 and Gereb et al. 2015, though we have 10 Associated spectra with redshift $z > 0.25$ reaching till $z = 1.2$. Whereas our intervening spectra have (mostly) $z >0.25$ (There are very few, around 3-4 with $z < 0.25$).

\iffalse
\begin{info} % Information block
	This is an interesting piece of information, to which the reader should pay special attention. Fusce varius orci ac magna dapibus porttitor. In tempor leo a neque bibendum sollicitudin. Nulla pretium fermentum nisi, eget sodales magna facilisis eu. Praesent aliquet nulla ut bibendum lacinia. Donec vel mauris vulputate, commodo ligula ut, egestas orci. Suspendisse commodo odio sed hendrerit lobortis. Donec finibus eros erat, vel ornare enim mattis et.
\end{info}
\fi
%----------------------------------------------------------------------------------------
%	PROBLEM 1
%----------------------------------------------------------------------------------------

\section*{Busy Function} % Numbered section

I will attach the image generated from busy function, with the free parameters mentioned in the figure \ref{fig:1}.

\begin{equation} \label{eq1}
\begin{aligned}
    B_1(x) = \frac{a}{4}\left( \text{erf}[b_1\{w + x - x_e\}] + 1\right) \left( \text{erf}[b_2\{w - x + x_e\}] + 1\right) \left( c|x - x_p|^n + 1 \right)
\end{aligned}
\end{equation}

Also note that the busyfit software also gives parameters like line centroid, half width, 20\% width, Peak flux and integrated flux for the spectra.

\begin{figure}[h]%
    \centering
    \subfloat[\centering Busy Function]{{\includegraphics[width=8cm]{busyfunction.png} }}%
    \qquad
    \subfloat[\centering Busy Function with spectra]{{\includegraphics[width=7cm]{Associated_94.png} }}%
    \caption{Busy function plots with parameters and spectra respectively. (The spectra used is an Associated spectra)}%
    \label{fig:1}%
\end{figure}
%------------------------------------------------
\section*{ROC Curves along with ROC AUC, Accuracy and Precision results with the removal of Redshift dependant parameters}

I have removed all the redshift dependant prameters like line widths and centroid position in the spectrum were removed and observed the following results

\begin{table}[h] 
  \centering
  \begin{tabular}{@{}cc@{}cc@{}cc@{}}
    \toprule
    ML Model & ROC AUC  | & Average Accuracy & Average Precision  \\
    \midrule
    Random Forest & $0.91$ & $86.8\%$ & $81.8\%$ \\
    KNN & $0.72$ & $79\%$ & $60.63\%$ \\
    Decision Tree & $0.799$ & $84.21\%$ & $56.6\%$ \\
    Logistic Regression & $0.78$ & $85.2\%$ & $73\%$ \\
    SVM & $0.698$ & $84.15\%$ & $64.6\%$ \\
    \bottomrule
  \end{tabular}
  \caption{Results for Machine Learning models after removing redshift dependant parameters.}
  \label{results}
\end{table}

\begin{info} % Information block
	These results are obtained using SMOTE to overcome class imbalence. ROC AUC is the area under the ROC curve. Also it should be noted that these are all average value of 30 folds i.e The data is split into 10 sets three times differently.
\end{info}

\iffalse
% Numbered question, with subquestions in an enumerate environment
\begin{question}
	Quisque ullamcorper placerat ipsum. Cras nibh. Morbi vel justo vitae lacus tincidunt ultrices. Lorem ipsum dolor sit amet, consectetuer adipiscing elit.

	% Subquestions numbered with letters
	\begin{enumerate}[(a)]
		\item Do this.
		\item Do that.
		\item Do something else.
	\end{enumerate}
\end{question}
\fi

I am also attaching the ROC plots for all the Machine Learning Models in Figure \ref{fig:2} (Next Page)

\begin{info} % Information block
	In the below plots, you will notice a lot of curves, the mean ROC curve for all the 30 folds is the thick blue curve. The other curves (thin) are for each fold validation, I haven't mentioned each legend as 30 legends won't fit. The plots also contain  $\pm 1 \sigma$ (Standard Deviation).
\end{info}


\begin{figure}[h]%
    \centering
    \subfloat[\centering Random Forest]{{\includegraphics[width=7.5cm]{ROC_curve_random_forest_smote.png} }}%
    \qquad
    \subfloat[\centering KNN]{{\includegraphics[width=7.5cm]{ROC_curve_knn_smote.png} }}%
    \qquad
    \subfloat[\centering Decision Tree]{{\includegraphics[width=7.5cm]{ROC_curve_decision_tree_smote.png} }}%
    \qquad
    \subfloat[\centering Logistic Regression]{{\includegraphics[width=7.5cm]{ROC_curve_logistic_regression_smote.png} }}%
    \qquad
    \subfloat[\centering SVM]{{\includegraphics[width=7.5cm]{ROC_curve_SVM_smote.png} }}%
    \caption{Mean ROC curve plots for all models}%
    \label{fig:2}%
\end{figure}

From the figures, we can note a decrease in ROC AUC, Accuracy and Precision. This can explained that the parameters which were redshift dependant played a major role in differentating the spectra into associated and intervening as the spectra does not have much overlap over the redshift range. Another reason can also be because we removed the differentiating factor (width) for associated and intervening spectra, as intervening spectra do have narrow line width compared to associated spectra spectra (Gupta et al. 2009 and Holt et al. 2008). Though width plays an important role, it still has high redshift dependance, either way we have a high redshift involment in the parameters. 

%------------------------------------------------
\iffalse
\subsection{Algorithmic issues}

In malesuada ullamcorper urna, sed dapibus diam sollicitudin non. Donec elit odio, accumsan ac nisl a, tempor imperdiet eros. Donec porta tortor eu risus consequat, a pharetra tortor tristique. Morbi sit amet laoreet erat. Morbi et luctus diam, quis porta ipsum. Quisque libero dolor, suscipit id facilisis eget, sodales volutpat dolor. Nullam vulputate interdum aliquam. Mauris id convallis erat, ut vehicula neque. Sed auctor nibh et elit fringilla, nec ultricies dui sollicitudin. Vestibulum vestibulum luctus metus venenatis facilisis. Suspendisse iaculis augue at vehicula ornare. Sed vel eros ut velit fermentum porttitor sed sed massa. Fusce venenatis, metus a rutrum sagittis, enim ex maximus velit, id semper nisi velit eu purus.

\begin{center}
	\begin{minipage}{0.5\linewidth} % Adjust the minipage width to accomodate for the length of algorithm lines
		\begin{algorithm}[H]
			\KwIn{$(a, b)$, two floating-point numbers}  % Algorithm inputs
			\KwResult{$(c, d)$, such that $a+b = c + d$} % Algorithm outputs/results
			\medskip
			\If{$\vert b\vert > \vert a\vert$}{
				exchange $a$ and $b$ \;
			}
			$c \leftarrow a + b$ \;
			$z \leftarrow c - a$ \;
			$d \leftarrow b - z$ \;
			{\bf return} $(c,d)$ \;
			\caption{\texttt{FastTwoSum}} % Algorithm name
			\label{alg:fastTwoSum}   % optional label to refer to
		\end{algorithm}
	\end{minipage}
\end{center}

Fusce varius orci ac magna dapibus porttitor. In tempor leo a neque bibendum sollicitudin. Nulla pretium fermentum nisi, eget sodales magna facilisis eu. Praesent aliquet nulla ut bibendum lacinia. Donec vel mauris vulputate, commodo ligula ut, egestas orci. Suspendisse commodo odio sed hendrerit lobortis. Donec finibus eros erat, vel ornare enim mattis et.

% Numbered question, with an optional title
\begin{question}[\itshape (with optional title)]
	In congue risus leo, in gravida enim viverra id. Donec eros mauris, bibendum vel dui at, tempor commodo augue. In vel lobortis lacus. Nam ornare ullamcorper mauris vel molestie. Maecenas vehicula ornare turpis, vitae fringilla orci consectetur vel. Nam pulvinar justo nec neque egestas tristique. Donec ac dolor at libero congue varius sed vitae lectus. Donec et tristique nulla, sit amet scelerisque orci. Maecenas a vestibulum lectus, vitae gravida nulla. Proin eget volutpat orci. Morbi eu aliquet turpis. Vivamus molestie urna quis tempor tristique. Proin hendrerit sem nec tempor sollicitudin.
\end{question}

Mauris interdum porttitor fringilla. Proin tincidunt sodales leo at ornare. Donec tempus magna non mauris gravida luctus. Cras vitae arcu vitae mauris eleifend scelerisque. Nam sem sapien, vulputate nec felis eu, blandit convallis risus. Pellentesque sollicitudin venenatis tincidunt. In et ipsum libero. Nullam tempor ligula a massa convallis pellentesque.

%----------------------------------------------------------------------------------------
%	PROBLEM 2
%----------------------------------------------------------------------------------------

\section{Implementation}

Proin lobortis efficitur dictum. Pellentesque vitae pharetra eros, quis dignissim magna. Sed tellus leo, semper non vestibulum vel, tincidunt eu mi. Aenean pretium ut velit sed facilisis. Ut placerat urna facilisis dolor suscipit vehicula. Ut ut auctor nunc. Nulla non massa eros. Proin rhoncus arcu odio, eu lobortis metus sollicitudin eu. Duis maximus ex dui, id bibendum diam dignissim id. Aliquam quis lorem lorem. Phasellus sagittis aliquet dolor, vulputate cursus dolor convallis vel. Suspendisse eu tellus feugiat, bibendum lectus quis, fermentum nunc. Nunc euismod condimentum magna nec bibendum. Curabitur elementum nibh eu sem cursus, eu aliquam leo rutrum. Sed bibendum augue sit amet pharetra ullamcorper. Aenean congue sit amet tortor vitae feugiat.

In congue risus leo, in gravida enim viverra id. Donec eros mauris, bibendum vel dui at, tempor commodo augue. In vel lobortis lacus. Nam ornare ullamcorper mauris vel molestie. Maecenas vehicula ornare turpis, vitae fringilla orci consectetur vel. Nam pulvinar justo nec neque egestas tristique. Donec ac dolor at libero congue varius sed vitae lectus. Donec et tristique nulla, sit amet scelerisque orci. Maecenas a vestibulum lectus, vitae gravida nulla. Proin eget volutpat orci. Morbi eu aliquet turpis. Vivamus molestie urna quis tempor tristique. Proin hendrerit sem nec tempor sollicitudin.

% File contents
\begin{file}[hello.py]
\begin{lstlisting}[language=Python]
#! /usr/bin/python

import sys
sys.stdout.write("Hello World!\n")
\end{lstlisting}
\end{file}

Fusce eleifend porttitor arcu, id accumsan elit pharetra eget. Mauris luctus velit sit amet est sodales rhoncus. Donec cursus suscipit justo, sed tristique ipsum fermentum nec. Ut tortor ex, ullamcorper varius congue in, efficitur a tellus. Vivamus ut rutrum nisi. Phasellus sit amet enim efficitur, aliquam nulla id, lacinia mauris. Quisque viverra libero ac magna maximus efficitur. Interdum et malesuada fames ac ante ipsum primis in faucibus. Vestibulum mollis eros in tellus fermentum, vitae tristique justo finibus. Sed quis vehicula nibh. Etiam nulla justo, pellentesque id sapien at, semper aliquam arcu. Integer at commodo arcu. Quisque dapibus ut lacus eget vulputate.

% Command-line "screenshot"
\begin{commandline}
	\begin{verbatim}
		$ chmod +x hello.py
		$ ./hello.py

		Hello World!
	\end{verbatim}
\end{commandline}

Vestibulum sodales orci a nisi interdum tristique. In dictum vehicula dui, eget bibendum purus elementum eu. Pellentesque lobortis mattis mauris, non feugiat dolor vulputate a. Cras porttitor dapibus lacus at pulvinar. Praesent eu nunc et libero porttitor malesuada tempus quis massa. Aenean cursus ipsum a velit ultricies sagittis. Sed non leo ullamcorper, suscipit massa ut, pulvinar erat. Aliquam erat volutpat. Nulla non lacus vitae mi placerat tincidunt et ac diam. Aliquam tincidunt augue sem, ut vestibulum est volutpat eget. Suspendisse potenti. Integer condimentum, risus nec maximus elementum, lacus purus porta arcu, at ultrices diam nisl eget urna. Curabitur sollicitudin diam quis sollicitudin varius. Ut porta erat ornare laoreet euismod. In tincidunt purus dui, nec egestas dui convallis non. In vestibulum ipsum in dictum scelerisque.

% Warning text, with a custom title
\begin{warn}[Notice:]
  In congue risus leo, in gravida enim viverra id. Donec eros mauris, bibendum vel dui at, tempor commodo augue. In vel lobortis lacus. Nam ornare ullamcorper mauris vel molestie. Maecenas vehicula ornare turpis, vitae fringilla orci consectetur vel. Nam pulvinar justo nec neque egestas tristique. Donec ac dolor at libero congue varius sed vitae lectus. Donec et tristique nulla, sit amet scelerisque orci. Maecenas a vestibulum lectus, vitae gravida nulla. Proin eget volutpat orci. Morbi eu aliquet turpis. Vivamus molestie urna quis tempor tristique. Proin hendrerit sem nec tempor sollicitudin.
\end{warn}
\fi
%----------------------------------------------------------------------------------------

\end{document}